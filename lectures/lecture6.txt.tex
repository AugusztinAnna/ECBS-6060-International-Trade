\documentclass[compress,mathserif,aspectratio=169]{beamer}
\usepackage[latin2]{inputenc}
%\usepackage[absolute]{textpos}
%\documentclass[handout,compress,mathserif]{beamer}
%\setbeameroption{show notes}

% This file is a solution template for:

% - Talk at a conference/colloquium.
% - Talk length is about 20min.
% - Style is ornate.



% Copyright 2004 by Till Tantau <tantau@users.sourceforge.net>.
%
% In principle, this file can be redistributed and/or modified under
% the terms of the GNU Public License, version 2.
%
% However, this file is supposed to be a template to be modified
% for your own needs. For this reason, if you use this file as a
% template and not specifically distribute it as part of a another
% package/program, I grant the extra permission to freely copy and
% modify this file as you see fit and even to delete this copyright
% notice.


\mode<presentation>
{
%  \usetheme{pittsburgh}
  % or ...

  \setbeamercovered{invisible}
  % or whatever (possibly just delete it)
}


\usepackage[USenglish]{babel}
%\usepackage[latin1]{inputenc}
%\usepackage[T1]{fontenc}
\usepackage{ifthen,array}



\pretolerance5000 \hyphenpenalty9999
%\setlength{\TPHorizModule}{0.5cm} \setlength{\TPVertModule}{0.5cm}
%\textblockorigin{20mm}{20mm} % start everything near the top-left corner

\newcounter{ora}
\newcounter{perc}
\newcounter{kezdoora}
\newcounter{kezdoperc}
\newcounter{percek}
\setcounter{percek}{15}
\setcounter{kezdoora}{4} % for 1.35pm as the starting time

\providecommand{\leadingzero}[1]{\ifthenelse{\value{#1}<10}{0\arabic{#1}}{\arabic{#1}}}
\providecommand{\oradisplay}[1]{\ifthenelse{\value{#1}<60}{\arabic{kezdoora}:\leadingzero{#1}}{\setcounter{perc}{\value{#1}}\addtocounter{perc}{-60}\setcounter{ora}{\value{kezdoora}}\addtocounter{ora}{1}\arabic{ora}:\leadingzero{perc}}}

\providecommand{\notes}[1]{{\tiny\textbf{Note:} #1}}
%%%%%%%%%%%%%%%%%%%%%%%%%%%%%%%%%%%%%%%%%%%%%%%%
%% Hasznos matek makrok
%%%%%%%%%%%%%%%%%%%%%%%%%%%%%%%%%%%%%%%%%%%%%%%%

\newcommand{\QED}{{}\hfill$\Box$}
\newcommand{\intl}[4]{\int_{#1}^{#2} \! {#3} \, \mathrm d{#4}}
\newcommand{\period}{\text{.}} % Ez azert kell, mert a matek . mashogy nez ki, mint a szovege.
\newcommand{\comma}{\text{,}}  % Ez azert kell, mert a matek , mashogy nez ki, mint a szovege.
\newcommand{\dist}{\,\mathop{\operatorname{\sim\,}}\limits}
\newcommand{\D}{\,\mathop{\operatorname{d}}\!}
%\newcommand{\E}{\mathop{\operatorname{E}}\nolimits}
\newcommand{\Lag}{\mathop{\operatorname{L}}}
\newcommand{\plim}{\mathop{\operatorname{plim}}\limits_{T\to\infty}\,}
\newcommand{\CES}[3]{\mathop{\operatorname{CES}}\left(\left\{#1\right\},\left\{#2\right\},#3\right)}
\newcommand{\cestwo}[5]{\left[#1^\frac1{#5}\,#2^\frac{#5-1}{#5}+#3^\frac1{#5}\,#4^\frac{#5-1}{#5}\right]^\frac{#5}{#5-1}}
\newcommand{\cesmore}[4]{\left[\sum_{#3}#1_{#3}^\frac1{#4}\,{#2}_{#3}^\frac{#4-1}{#4}\right]^\frac{#4}{#4-1}}
\newcommand{\cesPtwo}[5]{\left[#1\,#2^{1-#5}+#3\,#4^{1-#5}\right]^\frac{1}{1-#5}}
\newcommand{\cesPmore}[4]{\left[\sum_{#3}#1_{#3}\,#2_{#3}^{1-#4}\right]^\frac{1}{1-#4}}
\newcommand{\diff}[2]{\frac{\D #1}{\D #2}}
\newcommand{\pdiff}[2]{\frac{\partial #1}{\partial #2}}
\newcommand{\convex}[2]{\lambda #1 + (1-\lambda)#2}
\newcommand{\ABS}[1]{\left| #1 \right|}
\newcommand{\suchthat}{:\hskip1em}
\newcommand{\dispfrac}[2]{\frac{\displaystyle #1}{\displaystyle #2}} % Emeletes tortekhez hasznos.

\newcommand{\diag}{\mathop{\mathrm{diag\mathstrut}}}
\newcommand{\tr}{\mathop{\mathrm{tr\mathstrut}}}
\newcommand{\E}{\mathop{\mathrm{E\mathstrut}}}
\newcommand{\Var}{\mathop{\mathrm{Var\mathstrut}}\nolimits}
\newcommand{\Cov}{\mathop{\mathrm{Cov\mathstrut}}}
\newcommand{\sgn}{\mathop{\operatorname{sgn\mathstrut}}}

\newcommand{\covmat}{\mathbf\Sigma}
\newcommand{\ones}{\mathbf 1}
\newcommand{\zeros}{\mathbf 0}
\newcommand{\BAR}[1]{\overline{#1}}

\renewcommand{\time}[1]{\addtocounter{percek}{#1}}

\newlength{\tempsep}

\newenvironment{subeqs}{\setlength{\tempsep}{\arraycolsep}
\setlength{\arraycolsep}{0.13889em} % Ez azert kell, hogy ne hagyjon tul sok helyet az = korul.
\begin{subequations}\begin{eqnarray}}
{\end{eqnarray}\end{subequations}
\setlength{\arraycolsep}{\tempsep}}

\newenvironment{tapad}{\setlength{\tempsep}{\arraycolsep}
\setlength{\arraycolsep}{0.13889em}} % Ez azert kell, hogy ne hagyjon tul sok helyet az = korul.
{\setlength{\arraycolsep}{\tempsep}}

\newenvironment{eqnarr}{\setlength{\tempsep}{\arraycolsep}
\setlength{\arraycolsep}{0.13889em} % Ez azert kell, hogy ne hagyjon tul sok helyet az = korul.
\begin{eqnarray}}
{\end{eqnarray} \setlength{\arraycolsep}{\tempsep}}

\newenvironment{eqnarr*}{\setlength{\tempsep}{\arraycolsep}
\setlength{\arraycolsep}{0.13889em} % Ez azert kell, hogy ne hagyjon tul sok helyet az = korul.
\begin{eqnarray*}}
{\end{eqnarray*} \setlength{\arraycolsep}{\tempsep}}


%\usepackage[active]{srcltx} % SRC Specials: DVI [Inverse] Search
% Fuzz --- -------------------------------------------------------
\hfuzz5pt % Don't bother to report over-full boxes < 5pt
\vfuzz5pt % Don't bother to report over-full boxes < 5pt
% THEOREMS -------------------------------------------------------
% MATH -----------------------------------------------------------
\newcommand{\norm}[1]{\left\Vert#1\right\Vert}
\newcommand{\abs}[1]{\left\vert#1\right\vert}
\newcommand{\set}[1]{\left\{#1\right\}}
\newcommand{\Real}{\mathbb R}
\newcommand{\eps}{\varepsilon}
\newcommand{\To}{\longrightarrow}
\newcommand{\BX}{\mathbf{B}(X)}
\newcommand{\A}{\mathcal{A}}




\newcommand{\directory}{figures}
\newcommand*{\newtitle}{\egroup\begin{frame}\frametitle}

\newcommand{\fullpagefigure}[2]{\begin{frame}\frametitle{\hyperlink{#1back}{#2}}\hypertarget{#1}{{\begin{center}\includegraphics[height=0.9\textheight]{\directory/#1}\end{center}}}\end{frame}}
\newcommand{\widefigure}[2]{\begin{frame}\frametitle{\hyperlink{#1back}{#2}}\hypertarget{#1}{{\begin{center}\includegraphics[width=\linewidth]{\directory/#1}\end{center}}}\end{frame}}
\newcommand{\longfigure}[2]{\begin{frame}\frametitle{\hyperlink{#1back}{#2}}\hypertarget{#1}{{\begin{center}\includegraphics[height=0.8\textheight]{\directory/#1}\end{center}}}\end{frame}}
%\newcommand{\fullpagefigure}[2]{\begin{frame}\frametitle{\hyperlink{#1back}{#2}}\hypertarget{#1}{{\begin{centering}$#1$\end{centering}}}\end{frame}}
\newcommand{\answer}[1]{\begin{itemize}\item #1\end{itemize}}


\newcommand{\jumpto}[2]{\hypertarget{#1back}{\hyperlink{#1}{#2}}}
\newcommand{\backto}[2]{\hypertarget{#1}{\hyperlink{#1back}{#2}}}


\title{ECBS 6060: International Trade\\
Winter 2020}

\author{Mikl\'os Koren\\
korenm@ceu.edu}
% - Give the names in the same order as the appear in the paper.
% - Use the \inst{?} command only if the authors have different
%   affiliation.


\date % (optional, should be abbreviation of conference name)
{}
% - Either use conference name or its abbreviation.
% - Not really informative to the audience, more for people (including
%   yourself) who are reading the slides online

%\subject{Theoretical Computer Science}
% This is only inserted into the PDF information catalog. Can be left
% out.



% If you have a file called "university-logo-filename.xxx", where xxx
% is a graphic format that can be processed by latex or pdflatex,
% resp., then you can add a logo as follows:

\pgfdeclareimage[height=0.5cm]{university-logo}{frblogo}
%\logo{\pgfuseimage{university-logo}}



% Delete this, if you do not want the table of contents to pop up at
% the beginning of each subsection:
\AtBeginSection[]
{
  \begin{frame}[plain]
    \frametitle{\color{red}\insertsection}
    \addtocounter{framenumber}{-1}
    %\tableofcontents[currentsection,currentsubsection]
  \end{frame}
}


% If you wish to uncover everything in a step-wise fashion, uncomment
% the following command:

%\beamerdefaultoverlayspecification{<+->}

\setbeamertemplate{navigation symbols}{}
\setbeamertemplate{footline}{{}\hfill\insertframenumber}


\begin{document}

\begin{frame}[plain]
  \titlepage
    \addtocounter{framenumber}{-1}
\end{frame}






\section{Endowment-based theories}\hypertarget{Endowment-based theories}{}






\section{The integrated equilibrium}\hypertarget{The integrated equilibrium}{}
\begin{frame}\frametitle{The integrated equilibrium}\hypertarget{The integrated equilibrium}{}
\begin{itemize}
\item To emphasize factor proportions, we assume away all other differences:
\begin{enumerate}\setcounter{enumi}{0}
\item Preferences are identical and homothetic.

\item Technologies are the same.
\end{enumerate}

\item There are many countries, only differing in their factor endowments.

\item (What is the difference between factors and goods?)


\end{itemize}
\end{frame}



\begin{frame}\frametitle{The integrated equilibrium}\hypertarget{The integrated equilibrium}{}
\begin{itemize}
\item The \emph{\only<1>{integrated}\only<2>{John Lennon} equilibrium} is a useful benchmark:
\begin{itemize}
\item The equilibrium of the world economy where both goods and factors are mobile.
\end{itemize}

\item We derive trade patterns in the IE.

\item And study conditions for its existence.




\end{itemize}
\end{frame}



\begin{frame}\frametitle{Setup}\hypertarget{Setup}{}
\begin{itemize}
\item World endowment of factor $n\in N$: $\bar V_n$.

\item Identical, CRS, quasi-concave production function of good $i\in I$. Unit cost function: $c_i(w)$.
\begin{itemize}
\item Unit factor requirements:
    \[a_{ni}(w)=\frac{\partial}{\partial w_n}c_i(w)\]
\end{itemize}

\item Identical, homothetic  preferences.
\begin{itemize}
\item Consumption share:
    \[\alpha_i(p) = \frac{\partial e(p)}{e(p) \partial p_i}\]
\end{itemize}

\item Perfect competition.


\end{itemize}
\end{frame}



\begin{frame}\frametitle{Conditions of IE}\hypertarget{Conditions of IE}{}
\begin{itemize}
\item Profit maximization. For all $i$,
\[
p_i \le c_i(w), \text{ with $=$ if }x_i>0.
\]

\item Factor market clearing. For all $n$,
\[
\sum_i a_{ni}(w)\Bar x_i = \Bar V_n
\]

\item Goods market clearing. For all $i$,
\[
\alpha_i(p) = \frac{\Bar x_i}{\sum_j p_j\Bar x_j}.
\]


\end{itemize}
\end{frame}



\begin{frame}\frametitle{Carving up the world}\hypertarget{Carving up the world}{}
\begin{itemize}
\item Divide the world into $J$ countries.

\item Country $j$ has endowment $\{V^j_n\}$.

\item Under what conditions can the IE sustained?

\item We need to put restrictions on the set of endowments, $\mathbf V$.


\end{itemize}
\end{frame}



\begin{frame}\frametitle{Why IE is a useful benchmark}\hypertarget{Why IE is a useful benchmark}{}
\begin{itemize}
\item If we can replicate the IE, all countries face the same good and factor prices.

\item Hence $a_{ni}^j=a_{ni}(w)$ and $\alpha_i^j = \alpha_i(p)$ for all country $j$.

\item Clearly, profit and utility maximization will continue to hold.

\item So will goods market clearing.

\item But can we fully employ all factors in each country?


\end{itemize}
\end{frame}



\begin{frame}\frametitle{Full employment in each country}\hypertarget{Full employment in each country}{}
\begin{itemize}
\item Factor markets clear in each country $j$:
\[
\sum_i a_{ni}(w)x_i^j = V_n^j\,\forall n.
\]

\item Are there $x_i^j$s such that this holds and the world produces the \emph{same amount} as in the IE
\[
\sum_j x_i^j = \Bar x_i\,\forall i?
\]






\end{itemize}
\end{frame}




% Figures 1.1 and 1.2 here
\longfigure{HK1-1}{Factor demands of total world output in each sector}
\widefigure{HK1-2}{The set that replicates the integrated equilibrium}


\begin{frame}\frametitle{Factor price equalization}\hypertarget{Factor price equalization}{}
\begin{itemize}
\item As long as endowments are not \emph{too different} across countries, we can replicate the integrated equilibrium even if factors cannot flow across borders.

\item This will equalize factor prices, so that factors do not \emph{want} to move.

\item In this equilibrium, trade flows \emph{substitute} for factor flows.


% nontraded goods?


\end{itemize}
\end{frame}







\section{Pattern of trade}\hypertarget{Pattern of trade}{}
\begin{frame}\frametitle{Pattern of trade}\hypertarget{Pattern of trade}{}
\begin{itemize}
\item We begin with the 2-factor, 2-good case.

\item In this case we have a sharp prediction:
\pause


\begin{block}{Heckscher--Ohlin theorem}\hypertarget{Heckscher--Ohlin theorem}{}
Each country exports the good that uses its abundant factor intensively.




\end{block}
\end{itemize}
\end{frame}



\begin{frame}\frametitle{Heuristic proof using the law of comparative advantage}\hypertarget{Heuristic proof using the law of comparative advantage}{}
\begin{itemize}
\item In the integrated equilibrium, goods prices and factor prices are the same in the two countries.

\item In autarky, the labor abundant country has lower relative wage than the capital abundant country.

\item The autarky price of the labor intensive good will be lower in the labor abundant country.

\item It will hence export the labor intensive good and import the capital intensive one.


\end{itemize}
\end{frame}




\widefigure{HK1-3}{The pattern of trade with 2 goods and 2 factors}
% Figure 1.3 here




\begin{frame}\frametitle{More goods than factors}\hypertarget{More goods than factors}{}
\begin{itemize}
\item If we have more goods than factors, the pattern of \emph{goods trade} is indeterminate.

\item Luckily, we can still pin down the \emph{factor content} of trade.
\begin{itemize}
\item How much labor, capital, land (etc) are embedded in net exports?


\end{itemize}

\end{itemize}
\end{frame}




\widefigure{HK1-4}{The pattern of trade with 3 goods and 2 factors}




\begin{frame}\frametitle{Supply side}\hypertarget{Supply side}{}
\begin{itemize}
\item Let $\mathbf A$ denote the matrix of $[a_{in}]$s.

\item Because technology is the same in each country and factor prices equalize, $\mathbf A$ is the same across countries.

\item The factor content of production is
\[
\mathbf A\mathbf X^j = \sum_{i\in I}a_{in}X_{i}^j = \mathbf V^j.
\]

\item The factor content of consumption is
\[
\mathbf A\mathbf C^j = \sum_{i\in I}a_{in}C_{i}^j \neq \mathbf V^j.
\]


\end{itemize}
\end{frame}



\begin{frame}\frametitle{Demand side}\hypertarget{Demand side}{}
\begin{itemize}
\item Preferences are homothetic and prices are the same.

\item The consumption basket is the same across countries:
\[
\mathbf C^j = \mathbf \alpha  Y^j.
\]

\item In world equilibrium, consumption equals production,
\[
\sum_{j\in J} \mathbf C^j = \mathbf \alpha \sum_{j\in J}Y^j = \Bar {\mathbf X}.
\]

\item Clearly,
\[
\mathbf C^j = \frac{Y^j}{\sum_{k\in J}Y^k}\sum_{k\in J} \mathbf C^k
\equiv s^j\Bar {\mathbf C}.
\]

\item The factor content of consumption:
\[
\mathbf A\mathbf C^j = s^j \mathbf A\Bar {\mathbf C} = s^j \mathbf A\Bar {\mathbf X} = s^j \Bar {\mathbf  V}.
\]


\end{itemize}
\end{frame}



\begin{frame}\frametitle{The Vanek equation}\hypertarget{The Vanek equation}{}
\begin{itemize}
\item The factor content of net exports,
\[
\mathbf F^j \equiv \mathbf A\mathbf T^j = \mathbf A(\mathbf X^j-\mathbf C^j) = \mathbf V^j-s^j\Bar {\mathbf V}.
\]
\begin{block}{The Heckscher-Ohlin-Vanek theorem}\hypertarget{The Heckscher-Ohlin-Vanek theorem}{}
Each country exports the services of its abundant factors.


\end{block}
\end{itemize}
\end{frame}



\begin{frame}\frametitle{Balanced trade}\hypertarget{Balanced trade}{}
\begin{itemize}
\item If trade is balanced,
\[
p\mathbf T^j = 0.
\]

\item This implies
\[
w\mathbf F^j = 0.
\]
Why?

\item That is, some elements of $\mathbf F^j$ are positive, others are negative.


\end{itemize}
\end{frame}



\begin{frame}\frametitle{Net factor exports and imports}\hypertarget{Net factor exports and imports}{}
\begin{itemize}
\item Rank factors such that
\[
\frac{V_1^j}{\Bar V_1} > \frac{V_2^j}{\Bar V_2} > ...>s^j>...>
\frac{V_N^j}{\Bar V_N}.
\]

\item The first group of factors (${V_n^j}/{\Bar V_n}>s^j$) is exported, the second group of factors is imported.


\end{itemize}
\end{frame}




\widefigure{HK1-4}{The pattern of trade with 3 goods and 2 factors}




\begin{frame}\frametitle{Discussion}\hypertarget{Discussion}{}
\begin{itemize}
\item The HOV theorem sounds very much like a pure exchange economy. If I have more coconuts and you have more bananas, I will sell you coconuts for bananas.

\item However, there were many non-trivial steps involved in deriving it.

\item Empirical tests amount to a joint test of all these assumptions.








\end{itemize}
\end{frame}







\end{document}